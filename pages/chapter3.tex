% Chapter 3

\chapter{其它格式}

\section{定理}
\begin{theorem}[(中文括号)]\label{theorem:example}
	这是一条跟随章编号的定理,引用如\cref{theorem:example}。
\end{theorem}
\begin{definition}[ (English bracket)]
	这是一条定义。
\end{definition}
\begin{lemma}
	这是一条引理。
\end{lemma}
\begin{corollary}
	这是一条推论。
\end{corollary}
\begin{proposition}
	这是一条性质。
\end{proposition}
\begin{example}
	这是一条例。
\end{example}
\begin{remark}
	这是一条注。
\end{remark}
\begin{proof}
	这是一个证明,末尾自动添加证明结束符。
\end{proof}

\section{引用}
使用 \concise{\cref} 命令进行引用,将会自动检测环境并添加相应的前缀。

\subsection{脚注}
这里引用脚注\footnote{脚注示例文字。}。

\subsection{引用文中小节}\label{subsec:example}
如引用\cref{subsec:example}。

\subsection{参考文献}
默认是GB/T 7714-2015中国参考文献标准,顺序编码制风格,这是一个参考文献引用的范例\cite{1979Prospect}。

引用多个文献,将引用标号中的多个文献序号按升序排列,若其中有3个以上的连续序号,则改用范围序号,例如\cite{1979Prospect,2010An,1989The}。

\section{代码}
使用lstlisting环境可以对代码进行格式化
\begin{lstlisting}[language = Python]
import numpy as np

a = np.zeros((2,2))
print(a)
\end{lstlisting}

\section{物理量与国际单位}
siunitx宏包大大简化了符号输入,论文写作中很大一部分是单位、数字。这个宏包设置了一些命令,\concise{\num}~命令可以输入我们想要的各种方式的数字形式,比如科学计数法~\num{.3e45x1}。而~\concise{\SI}~命令用来输出标准单位,比如~\SI{10}{Hz},更多用法与单位见宏包文档。